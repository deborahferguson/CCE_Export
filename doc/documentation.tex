% *======================================================================*
%  Cactus Thorn template for ThornGuide documentation
%  Author: Ian Kelley
%  Date: Sun Jun 02, 2002
%  $Header$
%
%  Thorn documentation in the latex file doc/documentation.tex
%  will be included in ThornGuides built with the Cactus make system.
%  The scripts employed by the make system automatically include
%  pages about variables, parameters and scheduling parsed from the
%  relevant thorn CCL files.
%
%  This template contains guidelines which help to assure that your
%  documentation will be correctly added to ThornGuides. More
%  information is available in the Cactus UsersGuide.
%
%  Guidelines:
%   - Do not change anything before the line
%       % START CACTUS THORNGUIDE",
%     except for filling in the title, author, date, etc. fields.
%        - Each of these fields should only be on ONE line.
%        - Author names should be separated with a \\ or a comma.
%   - You can define your own macros, but they must appear after
%     the START CACTUS THORNGUIDE line, and must not redefine standard
%     latex commands.
%   - To avoid name clashes with other thorns, 'labels', 'citations',
%     'references', and 'image' names should conform to the following
%     convention:
%       ARRANGEMENT_THORN_LABEL
%     For example, an image wave.eps in the arrangement CactusWave and
%     thorn WaveToyC should be renamed to CactusWave_WaveToyC_wave.eps
%   - Graphics should only be included using the graphicx package.
%     More specifically, with the "\includegraphics" command.  Do
%     not specify any graphic file extensions in your .tex file. This
%     will allow us to create a PDF version of the ThornGuide
%     via pdflatex.
%   - References should be included with the latex "\bibitem" command.
%   - Use \begin{abstract}...\end{abstract} instead of \abstract{...}
%   - Do not use \appendix, instead include any appendices you need as
%     standard sections.
%   - For the benefit of our Perl scripts, and for future extensions,
%     please use simple latex.
%
% *======================================================================*
%
% Example of including a graphic image:
%    \begin{figure}[ht]
% 	\begin{center}
%    	   \includegraphics[width=6cm]{MyArrangement_MyThorn_MyFigure}
% 	\end{center}
% 	\caption{Illustration of this and that}
% 	\label{MyArrangement_MyThorn_MyLabel}
%    \end{figure}
%
% Example of using a label:
%   \label{MyArrangement_MyThorn_MyLabel}
%
% Example of a citation:
%    \cite{MyArrangement_MyThorn_Author99}
%
% Example of including a reference
%   \bibitem{MyArrangement_MyThorn_Author99}
%   {J. Author, {\em The Title of the Book, Journal, or periodical}, 1 (1999),
%   1--16. {\tt http://www.nowhere.com/}}
%
% *======================================================================*

% If you are using CVS use this line to give version information
% $Header$

\documentclass{article}

% Use the Cactus ThornGuide style file
% (Automatically used from Cactus distribution, if you have a
%  thorn without the Cactus Flesh download this from the Cactus
%  homepage at www.cactuscode.org)
\usepackage{../../../../doc/latex/cactus}

\begin{document}

% The author of the documentation
\author{Deborah Ferguson}

% The title of the document (not necessarily the name of the Thorn)
\title{CCE_Export}

% the date your document was last changed, if your document is in CVS,
% please use:
%    \date{$ $Date$ $}
% when using git instead record the commit ID:
%    \date{\gitrevision{<path-to-your-.git-directory>}}
\date{\gitrevision{<path-to-your-.git-directory>}}

\maketitle

% Do not delete next line
% START CACTUS THORNGUIDE

% Add all definitions used in this documentation here
%   \def\mydef etc

% Add an abstract for this thorn's documentation
\begin{abstract}

    Outputs EinsteinToolkit data in a format which can be read by SpECTRE's CCE code for obtaining waveform data at infinity.

\end{abstract}

% The following sections are suggestive only.
% Remove them or add your own.

\section{Introduction}

This thorn allows the user to output the metric, lapse, and shift data in a format which can be fed directly into SpECTRE's
Cauchy Characteristic Evolution code (documentation for that code here: https://spectre-code.org/tutorial_cce.html).
Doing so enables the evolution of gravitational radiation to spatial and temporal infinity as well as the direct extraction
of gravitational wave strain.

\section{Numerical Implementation}

The data is output into an hdf5 file for each radius. Within this h5 file are datasets for the metric, lapse, shift, and
their respective radial and time derivatives:

gxx.dat, gxy.dat, gxz.dat, gyy.dat, gyz.dat, gzz.dat
Drgxx.dat, Drgxy.dat, Drgxz.dat, Drgyy.dat, Drgyz.dat, Drgzz.dat
Dtgxx.dat, Dtgxy.dat, Dtgxz.dat, Dtgyy.dat, Dtgyz.dat, Dtgzz.dat
Shiftx.dat, Shifty.dat, Shiftz.dat
DrShiftx.dat, DrShifty.dat, DrShiftz.dat
DtShiftx.dat, DtShifty.dat, DtShiftz.dat
Lapse.dat
DrLapse.dat
DtLapse.dat

Each quantity is decomposed using spin weight 0 spherical harmonics. The first column denotes the time, and the remaining
columns are the real and imaginary coefficients for the modes in m-changes-fastest order:

"time", "Re(0,0)", "Im(0,0)", "Re(1,-1)", "Im(1,-1)", "Re(1,0)", "Im(1,0)",
"Re(1,1)", "Im(1,1)", "Re(2,-2)", "Im(2,-2)", "Re(2,-1)", "Im(2,-1)", "Re(2,0)",
"Im(2,0)", "Re(2,1)", "Im(2,1)", "Re(2,2)", "Im(2,2)", ...

Spherical harmonic convention: ?

The radial derivatives are computed by transforming the cartesian derivatives returned from the interpolater. 
The time derivatives of the lapse and shift are obtained as grid variables, and the time derivative of the metric
is computed from the extrinsic curvature. 

\section{Using This Thorn}

\subsection{Obtaining This Thorn}

This thorn can be obtained at https://github.com/deborahferguson/CCE_Export and is included as part of the EinsteinToolkit.

\subsection{Basic Usage}

The user specifies how many radii they would like to extract at (\texttt{nradii}) and the value of each of those radii 
in M (\texttt{radius\[\]}). The user should also specify how often they want to output the data in iterations (\texttt{out\_every}).
Optionally, the user can also provide a specific location to output the files (\texttt{out\_dir}) as well as the desired 
number of iterations to preallocate in the h5 files (\texttt{hdf5\_chunk\_size}).

\subsection{Interaction With Other Thorns}

The functionality of this thorn relies upon grid variables set up in ADMBase, specifically those in the groups metric, curv, 
lapse, shift, dtlapse, and dtshift. These grid variables must all be populated in order to use this thorn. Since some 
evolution codes do not write the time derivatives of the lapse and shift to these variables, those evolution codes are not 
compatible with this thorn.

\subsection{Examples}

CCE_Export::nradii       = 3
CCE_Export::radius[0]    = 100.00
CCE_Export::radius[1]    = 150.00
CCE_Export::radius[2]    = 200.00
CCE_Export::out_every    = 32

\subsection{Support and Feedback}

If any issues arise, please create a ticket on the github.


\begin{thebibliography}{9}

\end{thebibliography}

% Do not delete next line
% END CACTUS THORNGUIDE

\end{document}
